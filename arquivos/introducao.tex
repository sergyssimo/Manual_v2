\chapter{Introdução}
Este capítulo contém, entre outras coisas, informações importantes sobre a estrutura do manual e a história do \KOMAScript\, que começa anos antes da primeira versão. Você também encontrará informações sobre como instalar o \KOMAScript\ e o que fazer se encontrar erros.

\section{Nota preliminar}
O \KOMAScript\ é muito complexo. Isso se deve ao fato de que ele consiste não apenas em uma única classe ou um único pacote, mas em um conjunto de muitas classes e pacotes. Embora as classes sejam projetadas como contrapartes das classes padrão, isso não significa que elas forneçam apenas os comandos, ambientes e configurações das classes padrão, ou que imitem sua aparência. Os recursos do \KOMAScript\ às vezes superam em muito os das classes padrão. Alguns deles devem ser considerados extensões dos recursos básicos do kernel \LaTeX.

O exposto acima significa que a documentação do \KOMAScript\ deve ser extensa. Além disso, o \KOMAScript\ normalmente não é ensinado. Isso significa que não há professores que conheçam seus alunos e possam, portanto, escolher os materiais didáticos e adaptá-los adequadamente. Seria fácil escrever documentação para um público específico. A dificuldade enfrentada pelo autor, no entanto, é que o manual deve atender a todos os públicos potenciais. Tentei criar um guia que seja igualmente adequado para o cientista da computação e a secretária do peixeiro. Eu tentei, embora esta seja realmente uma tarefa impossível. O resultado são vários compromissos, e eu pediria que você levasse esse problema em consideração se tiver alguma reclamação ou sugestão para ajudar a melhorar a situação atual.

Apesar da extensão deste manual, eu pediria que você consultasse a documentação primeiro caso tenha problemas. Você deve começar consultando o índice de várias partes no final deste documento. Além deste manual, a documentação inclui todos os documentos de texto que fazem parte do pacote. Veja \texttt{manifest.tex} para uma lista completa.
\section{Estrutura do Guia}
Este manual é dividido em várias partes: Há uma seção para usuários comuns, uma para usuários avançados e especialistas, e um apêndice com mais informações e exemplos para aqueles que querem entender o \KOMAScript\ completamente.

A Parte I é destinada a todos os usuários do \KOMAScript. Isso significa que algumas informações nesta seção são direcionadas a novatos no \LaTeX. Em particular, esta parte contém muitos exemplos que visam esclarecer as explicações. Não hesite em experimentar esses exemplos você mesmo e descobrir como o \KOMAScript\ funciona modificando-os. Dito isso, o guia do usuário do \KOMAScript\ não pretende ser uma cartilha do \LaTeX. Aqueles que são novos no \LaTeX devem dar uma olhada em \textit{The Not So Short Introduction} ao \LaTeXe ou \LaTeXe para Autores ou um livro de referência \LaTeX. Você também encontrará informações úteis nas muitas FAQs do \LaTeX, incluindo as Perguntas Frequentes do \TeX\ na Web. Embora o tamanho das Perguntas Frequentes do \TeX\ na Web seja considerável, você deve obter pelo menos uma visão geral aproximada e consultá-la caso tenha problemas, bem como este guia.

A Parte II é destinada a usuários avançados do \KOMAScript\, aqueles que já estão familiarizados com \LaTeX\ ou que trabalham com o \KOMAScript\ há algum tempo e querem entender mais sobre como ele funciona, como ele interage com outros pacotes e como executar tarefas mais especializadas com ele. Para esse propósito, retornamos a alguns aspectos das descrições de classe da Parte I e os explicamos em mais detalhes. Além disso, documentamos alguns comandos que são particularmente destinados a usuários avançados e especialistas. Isto é complementado pela documentação de pacotes que normalmente são escondidos do usuário, na medida em que eles fazem seu trabalho abaixo da superfície das classes e pacotes do usuário. Esses pacotes são especificamente projetados para serem usados por autores de classes e pacotes.

O apêndice, que só pode ser encontrado na versão do livro em alemão, contém informações além daquelas que são cobertas na parte I e parte II. Usuários avançados encontrarão informações básicas sobre questões de tipografia para dar a eles uma base para suas próprias decisões. Além disso, o apêndice fornece exemplos para aspirantes a autores de pacotes. Esses exemplos não se destinam simplesmente a serem copiados. Em vez disso, eles fornecem informações sobre planejamento e implementação de projetos, bem como alguns comandos básicos do \LaTeX\ para autores de pacotes.

O layout do guia deve ajudá-lo a ler apenas as partes que são realmente de interesse. Cada classe e pacote normalmente tem seu próprio capítulo. Referências cruzadas para outro capítulo são, portanto, geralmente também referências a outra parte do pacote geral. No entanto, como as três principais classes (\texttt{scrbook}, \texttt{scrrprt} e \texttt{scrartcl}) concordam amplamente, elas são apresentadas juntas no capítulo 3. Diferenças entre as classes, por exemplo, para algo que afeta apenas a classe \texttt{scrartcl}, são claramente destacadas na margem, como mostrado aqui com \texttt{scrartcl}.

A documentação primária do \KOMAScript\ está em alemão e foi traduzida para sua conveniência; como a maioria do mundo \LaTeX, seus comandos, ambientes, opções, etc., estão em inglês. Em alguns casos, o nome de um comando pode soar um pouco estranho, mas mesmo assim, esperamos e acreditamos que com a ajuda deste guia, o \KOMAScript\ será utilizável e útil para você.

Neste ponto, você deve saber o suficiente para entender o guia. No entanto, ainda pode valer a pena ler o restante deste capítulo.

\section{História do \KOMAScript}
No início dos anos 1990, Frank Neukam precisava de um método para publicar as notas de aula de um instrutor. Naquela época, o \LaTeX\ era \LaTeXe\ e não havia distinção entre classes e pacotes — havia apenas estilos. Frank sentiu que os estilos de documentos padrão não eram bons o suficiente para seu trabalho; ele queria comandos e ambientes adicionais. Ao mesmo tempo, ele estava interessado em tipografia e, depois de ler\textit{ Ausgewählte Aufsätze über Fragen der Gestalt des Buches und der Typographie}, de Tschichold (Artigos selecionados sobre os problemas de design de livros e Tipografia), ele decidiu escrever seu próprio estilo de documento — e não apenas uma solução única para suas notas de aula, mas uma família de estilos inteira, uma projetada especificamente para tipografia europeia e alemã. Assim nasceu o Script.

Markus Kohm, o desenvolvedor do \KOMAScript\, encontrou o \textbf{Script} em dezembro de 1992 e adicionou uma opção para usar o formato de papel A5. Naquela época, nem o estilo padrão nem o Script forneciam suporte para papel A5. Portanto, não demorou muito para que Markus fizesse as primeiras alterações no Script. Essas e outras alterações foram então incorporadas ao \textbf{Script-2}, lançado por Frank em dezembro de 1993.

Em meados de 1994, o \LaTeXe\ ficou disponível e trouxe consigo muitas alterações. Usuários do Script-2 enfrentaram a opção de limitar seu uso ao modo de compatibilidade do \LaTeXe\ ou desistir do Script completamente. Essa situação levou Markus a montar um novo pacote \LaTeX2e, lançado em 7 de julho de 1994 como \KOMAScript\. Poucos meses depois, Frank declarou o \KOMAScript\ como o sucessor oficial do Script. O \KOMAScript\ originalmente não fornecia nenhuma classe \textit{letter}, mas essa deficiência foi logo corrigida por Axel Kielhorn, e o resultado se tornou parte do \KOMAScript\ em dezembro de 1994. Axel também escreveu o primeiro verdadeiro guia do usuário em alemão, que foi seguido por um guia em inglês de Werner Lemberg.

Desde então, muito tempo se passou. O\ LaTeX\ mudou apenas em pequenas coisas, mas o seu cenário mudou muito; muitos novos pacotes e classes estão agora disponíveis e o próprio \KOMAScript\ cresceu muito além do que era em 1994. O objetivo inicial era fornecer boas classes LaTeX para autores de língua alemã, mas hoje seu propósito principal é fornecer alternativas mais flexíveis às classes padrão. O sucesso do \KOMAScript\ levou a e-mails de usuários de todo o mundo, e isso levou a muitas novas macros — todas precisando de documentação; daí este “pequeno guia”.

\section{Agradecimentos especiais}
Agradecimentos na introdução? Não, os agradecimentos adequados podem ser encontrados no adendo. Meus comentários aqui não são destinados aos autores deste guia — e esses agradecimentos devem vir de você, o leitor, de qualquer forma. Eu, o autor do \KOMAScript, gostaria de estender meus agradecimentos pessoais a Frank Neukam. Sem sua família Script, \KOMAScript\ não teria surgido. Sou grato às muitas pessoas que contribuíram para o \KOMAScript\, mas com sua indulgência, gostaria de mencionar especificamente Jens-Uwe Morawski e Torsten Krüger. A tradução em inglês do guia é, entre muitas outras coisas, devido ao comprometimento incansável de Jens. Torsten foi o melhor testador beta que já tive. Seu trabalho melhorou particularmente a usabilidade do scrlttr2 e scrpage2. Muito obrigado a todos que me encorajaram a continuar, a tornar as coisas melhores e menos propensas a erros, ou a implementar recursos adicionais.

Agradecimentos especiais também aos fundadores e membros do DANTE, \textit{Deutschsprachige Anwendervereinigung} \TeX\ e.V, (Grupo de Usuários do \TeX\ em Língua Alemã). Sem o servidor DANTE, o \KOMAScript\ não poderia ter sido lançado e distribuído. Obrigado também a todos nos grupos de notícias e listas de discussão do \TeX\ que respondem perguntas e me ajudaram a dar suporte ao \KOMAScript.

Meus agradecimentos também vão para todos aqueles que sempre me encorajaram a ir além e implementar este ou aquele recurso melhor, com menos falhas, ou simplesmente como uma extensão. Eu também gostaria de agradecer ao doador muito generoso que me deu a quantia mais significativa de dinheiro que eu já recebi pelo trabalho feito até agora no \KOMAScript.
