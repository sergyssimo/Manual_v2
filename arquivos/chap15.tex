\chapter{Ambiente Quote/Quotation}
\begin{verbatim}
    \begin{quotation}...\end{quotation}  
\end{verbatim}
Esses dois ambientes também são definidos internamente como ambientes de lista e podem ser encontrados em ambas as classes \textbf{standard} e \KOMAScript. Ambos os ambientes usam texto justificado que é recuado tanto no lado esquerdo quanto no direito. Frequentemente, eles são usados para separar citações mais longas do texto principal. A diferença entre os dois está na maneira como os parágrafos são compostos. Enquanto os parágrafos de citação são distinguidos pelo espaço vertical, nos parágrafos de citação, a primeira linha é recuada. Isso também se aplica à primeira linha de um \texttt{quo\-te\-en\-vi\-ron\-ment}, a menos que seja precedido por \verb|\noindent|.
