\chapter{Cabeçalhos e Rodapés com \texttt{scrlayer-scrpage}}

Até a versão $3.11b$ do \KOMAScript\ o pacote \textbf{scrpage2} era a maneira recomendada de personalizar cabeçalhos e rodapés além das opções fornecidas pelos estilos de página headings, myheadings, plain e empty das classes \KOMAScript. Desde 2013, o pacote \textbf{scrlayer} foi incluído como um módulo básico do \KOMAScript. Este pacote fornece um modelo de camada e um novo modelo de estilo de página com base nele. No entanto, a interface do pacote é quase flexível demais e, consequentemente, não é fácil para o usuário médio compreender. Para obter mais informações sobre esta interface, consulte o capítulo 17 na parte II. No entanto, algumas das opções que realmente fazem parte do \textbf{scrlayer} e que, portanto, são retomadas naquele capítulo, também são documentadas aqui porque são necessárias para usar o \textbf{scrlayer-scrpage}.

Muitos usuários já estão familiarizados com os comandos do scrpage2. Por esse motivo, o scrlayer-scrpage fornece um método para manipular cabeçalhos e rodapés que é baseado no scrlayer, é amplamente compatível com o scrpage2 e, ao mesmo tempo, expande muito a interface do usuário. Se você já estiver familiarizado com o scrpage2 e se abster de chamadas diretas para seus comandos internos, normalmente pode usar o scrlayer-scrpage como um substituto imediato. Isso também se aplica à maioria dos exemplos usando o scrpage2 encontrados em livros do \LaTeX\ ou na Internet.

Além do scrlayer-scrpage ou scrpage2, você também pode usar o \textbf{fancyhdr} (veja [vO04]) para configurar os cabeçalhos e rodapés das páginas. No entanto, este pacote não tem suporte para vários recursos do \KOMAScript\ por exemplo, o esquema de elementos (veja \char`\\\texttt{set\-ko\-ma\-font}, \char`\\\texttt{add\-to\-ko\-ma\-font} e \char`\\\texttt{use\-ko\-ma\-font} na seção 3.6, página 58) ou o formato de numeração configurável para cabeçalhos dinâmicos (veja a opção numbers e, por exemplo, \char`\\\texttt{chap\-ter\-mark\-for\-mat} na seção 3.16, página 99 e página 112). Portanto, se você estiver usando uma classe \KOMAScript\ você deve usar o novo pacote scrlayer-scrpage. Se você tiver problemas, você ainda pode usar o scrpage2. Claro, você também pode usar o scrlayer-scrpage com outras classes, como as do \LaTeX\ padrão.

Além dos recursos descritos neste capítulo, o scrlayer-scrpage fornece várias outras funções que provavelmente interessam apenas a um número muito pequeno de usuários e, portanto, são descritas no capítulo 18 da parte II, começando na página 448. No entanto, se as opções descritas na parte I forem insuficientes para seus propósitos, você deve examinar o capítulo 18.

\section{Altura do Cabeçalho e Rodapé}
As classes padrão do \LaTeX\ não usam muito o rodapé e, se o usam, colocam o conteúdo em uma \cmd{mbox}, o que resulta no rodapé sendo uma única linha de texto. Esta é provavelmente a razão pela qual o próprio \LaTeX\ não tem uma altura de rodapé bem definida. Embora a distância entre a última linha de base da área de texto e a linha de base do rodapé seja definida com \cmd{footskip}, se o rodapé consistir em mais de uma linha de texto, não há uma declaração definitiva se esse comprimento deve ser a distância até a primeira ou a última linha de base do rodapé.

Embora o cabeçalho da página das classes padrão também seja colocado em uma caixa horizontal e, portanto, também seja uma única linha de texto, o \LaTeX\ na verdade fornece um comprimento para definir a altura do cabeçalho. A razão para isso pode ser que essa altura seja necessária para determinar o início da área de texto.
\begin{verbatim}
        \footheight
        \headheight
        autoenlargeheadfoot=simple switch
\end{verbatim}

O pacote scrlayer introduz um novo comprimento, \cmd{footheight}, análogo a \cmd{head\-hei\-ght}. Além disso, scrlayer-scrpage interpreta \cmd{footskip} como a distância da última linha de base da área de texto até a primeira linha de base normal do rodapé. O pacote \textbf{typearea} interpreta \cmd{footheight} da mesma forma, então as opções do \textbf{typearea} para a altura do rodapé também podem ser usadas para definir os valores para o pacote scrlayer. Veja as opções \texttt{footheight} e \texttt{footlines} na seção 2.6, página 45) e a opção \texttt{foot\-inclu\-de} na página 42 da mesma seção.

Se você não usar o pacote \textbf{typearea}, você deve ajustar as alturas do cabeçalho e rodapé usando valores apropriados para os comprimentos quando necessário. Para o cabeçalho, pelo menos, o pacote \textbf{geometry}, por exemplo, fornece configurações semelhantes.

Se você escolher uma altura de cabeçalho ou rodapé que seja muito pequena para o conteúdo real, scrlayer-scrpage tenta por padrão ajustar os comprimentos apropriadamente. Ao mesmo tempo, ele emitirá um aviso contendo sugestões para configurações adequadas. Essas alterações automáticas entram em vigor imediatamente após a necessidade delas ter sido detectada e não são revertidas automaticamente, por exemplo, quando o conteúdo do cabeçalho ou rodapé fica menor depois. No entanto, esse comportamento pode ser alterado usando a opção \texttt{auto\-en\-lar\-ge\-head\-foot}. Esta opção reconhece os valores para interruptores simples na tabela 2.5, página 41. A opção é ativada por padrão. Se for desativada, o cabeçalho e o rodapé não serão mais ampliados automaticamente. Apenas um aviso com dicas para configurações adequadas é emitido.


