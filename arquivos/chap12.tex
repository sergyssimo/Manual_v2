\chapter{Estrutura do Livro}
Às vezes, os livros são vagamente divididos em matéria frontal, matéria principal e matéria final. O \KOMAScript\ também fornece essa capacidade para \textbf{scrbook}.
\begin{verbatim}
        \frontmatter
        \mainmatter
        \backmatter
\end{verbatim}

A matéria frontal inclui todo o material que aparece antes do texto principal começar, incluindo páginas de título, prefácio e índice. Ele é iniciado com \verb|\frontmatter|. Na matéria frontal, algarismos romanos são usados para os números de página, e os títulos dos capítulos no cabeçalho não são numerados. No entanto, os títulos das seções são numerados consecutivamente, começando do capítulo 0. Isso normalmente não importa, pois a matéria frontal é usada apenas para as páginas de título, índice de conteúdo, listas de figuras e tabelas e um prefácio. O prefácio pode, portanto, ser criado como um capítulo normal. Um prefácio deve ser o mais curto possível e nunca dividido em seções. O prefácio, portanto, não requer um nível mais profundo de estrutura do que o capítulo.

Se você vê as coisas de forma diferente e deseja usar seções numeradas nos capítulos da matéria frontal, a partir da versão $2.97e$, a numeração da seção não contém mais o número do capítulo. Esta alteração só entra em vigor quando a opção de compatibilidade é definida para pelo menos a versão $2.97e$ (veja a opção versão, seção 3.2, página 55). É explicitamente notado que isso cria confusão com números de capítulos! O uso de \verb|\addsec| e \verb|\section*| (veja a seção 3.16, página 105 e página 106) são, portanto, na opinião do autor, muito preferíveis.

A partir da versão $2.97e$ a numeração de ambientes flutuantes, como tabelas e figuras, e números de equações no \textbf{front matter} também não contém nenhuma parte de número de capítulo. Para entrar em vigor, isso também requer a configuração de compatibilidade correspondente (veja a opção versão, seção 3.2, página 55).

A parte do livro com o texto principal é introduzida com \char`\\\texttt{main\-matter}. Se não houver \textbf{front matter}, você pode omitir este comando. A numeração de páginas padrão no \textbf{main matter} usa algarismos arábicos e (re)inicia a contagem de páginas em 1 no início do \textbf{main matter}.

O \textbf{back matter} é introduzido com \verb|\backmatter|. As opiniões diferem quanto ao que pertence ao \textbf{back matter}. Então, em alguns casos, você encontrará apenas a bibliografia, em outros, apenas o índice ou, ainda, ambos, bem como os apêndices. Os capítulos no\textbf{ back matter} são semelhantes aos capítulos no\textbf{ front matter}, mas a numeração de páginas não é redefinida. Se você precisar de numeração de páginas separada, pode usar o comando \cmd{pa\-ge\-num\-be\-ring} na seção 3.12, página 85.
