\chapter{Ambiente Addmargin}

\begin{verbatim}
\begin{addmargin}[left indentation ]{indentation }...\end{addmargin}
\begin{addmargin*}[inner indentation]{indentation }...\end{addmargin*}
\end{verbatim}

Assim como quote e \texttt{quotation}, o ambiente \texttt{addmargin} altera a margem. No entanto, diferentemente dos dois primeiros ambientes, \texttt{addmargin} permite que o usuário altere a largura do recuo. Além desta alteração, este ambiente não altera o recuo da primeira linha nem o espaçamento vertical entre parágrafos.

Se apenas o argumento obrigatório \texttt{indentation} for fornecido, as margens esquerda e direita serão expandidas por este valor. Se o argumento opcional \texttt{left indentation} também for fornecido, então o valor left indentation será usado para a margem esquerda em vez de \texttt{indentation}.

A variante com estrela \texttt{addmargin*} difere da versão normal apenas no modo de dois lados. Além disso, a diferença só ocorre se o argumento opcional \texttt{inner indentation} for usado. Neste caso, o valor de \texttt{inner indentation} é adicionado ao recuo interno normal. Para páginas à direita, esta é a margem esquerda; para páginas à esquerda, a margem direita. Então o valor de \texttt{indentation} determina a largura da margem oposta.

Ambas as versões deste ambiente permitem valores negativos para todos os parâmetros. Isso pode ser feito para que o ambiente se projete para a margem.
